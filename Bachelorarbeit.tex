
\input{preamble}
\makenoidxglossaries

\newacronym{bsm}{BSM}{Beyond the Standard Model}
\newacronym{eft}{EFT}{Effective Field Theory}
\newacronym{gm}{GM}{Georgi-Machacek Model}
\newacronym{lhc}{LHC}{Large Hadron Collider}
\newacronym{mssm}{MSSM}{Minimal Supersymmetric Standard Model}
\newacronym{qgc}{QGC}{Quadratic Gauge Coupling}
\newacronym{sm}{SM}{Standard Model}
\newacronym{smeft}{SMEFT}{Standard Model effective field theory}
\newacronym{susy}{SUSY}{Supersymmetry}
\newacronym{tgc}{TGC}{Triple Gauge Coupling}
\newacronym{uv}{UV}{Ultraviolet}
\newacronym{vbs}{VBS}{Vector Boson Scattering}
\addbibresource{bib.bib}
\pagenumbering{roman}
\begin{document}

% Titelpageseite
\begin{titlepage}
    \begin{tabularx}{\linewidth}{X}
        \includegraphics[width=6cm]{TU_Logo_SW} \\\hline\hline

        \vspace{4.5em}

        \begin{singlespace}\begin{center}\bfseries\Huge

                Sensitivity of dimension-8 EFT operators to resonance signals in Vector boson scattering

            \end{center}\end{singlespace}

        \vspace{5.5em}

        \begin{singlespace}\begin{center}\large
                Bachelor-Arbeit \\ zur Erlangung des Hochschulgrades \\
                Bachelor of Science \\
                im Bachelor-Studiengang Physik
            \end{center}\end{singlespace}\medskip

        \begin{center}vorgelegt von\end{center}
        \begin{center}
            {\large Georg Schmieder} \\ geboren am 14.10.1996 in Oschersleben
        \end{center}\medskip

        \begin{singlespace}\begin{center}\large
                Institut für Kern und Teilchen Physik \\
                Fakultät Physik \\
                Bereich Mathematik und Naturwissenschaften \\
                Technische Universität Dresden \\ 2022
            \end{center}\end{singlespace}
    \end{tabularx}
\end{titlepage}


% Gutachterseite
\thispagestyle{empty}\vspace*{48em}

Eingereicht am 10.~Mai~2022\vspace{1.5em}
\par{\large\begin{tabular}{ll}
        1. Gutachter: & Dr.~Frank Siegert         \\
        2. Gutachter: & Prof.~Dr.~Arno Straessner \\
    \end{tabular}}


% Abstractseite
\newpage
\begin{center}\large\bfseries Summary\end{center}


Abstract \\
English: \\
Experimental evidence has shown that the \acrfull{sm} is not complete theory.
Research for \acrfull{bsm} particles are conducted at the \acrfull{lhc}. With Run 2 \acrfull{vbs} processes gained importance in the analysis of \acrfull{tgc} and \acrfull{qgc}.
In these process new physics are theorized and several theories suggest particles with a mass lower than 1 TeV.
An \acrfull{eft} is introduced to describe a variety of different theories. In \acrshort{eft} an effective Lagrangian is derived containing operators for high mass dimensions.
In this thesis dim-8 \acrshort{eft} operators are investigated in the scattering of a $W^{\pm}$ and a $Z$ boson. The dim-8 operators are fitted against resonances in the range from 225 GeV to 1000 GeV.
For studying the sensitivity the invariant mass and transverse mass are analyzed. The objective is to find the lowest resonances mass that can be described using dim-8 \acrshort{eft} operators and setting the EFT limits for these resonances.


\vspace{10em}
Abstract \\
Deutsch: \\
Experimente haben gezeigt das, dass Standardmodell keine vollständige Theory ist.
Am Large Hadron Collider wird nach Teilchen gesucht, die das Standardmodell erweitern.
Mit Run 2 wurden Vektorbosonen Streuungen (VBS) wichtiger da sich einfach dreifach und vierfach Kopplungen untersuchen lassen.
Diese Prozesse werden als gute Kandidaten für neue Physik gesehen und moderne Theorien schlagen neue Teilchen mit Masse kleiner als 1 TeV vor.
Es können Effektive Feld Theorien (EFT) eingeführt werden, die eine große Anzahl dieser Theorien beschreiben können.
In EFT wird eine effektive Lagrange Funktion definier die Operatoren mit hohen Masse Dimensionen beinhalten.
In dieser Arbeit werden dim-8 EFT Operatoren in der Streuung von $W{\pm}$ und $Z$ Bosonen untersucht. Die dim-8 Operatoren werden an Resonanzen gefittet mit Massen zwischen 225 GeV und 1000 GeV.
Die Sensitivität wir für die invariante Masse und die transversale Masse analysiert. Mit dem Zeil die niedrigste Masse,
die mit dim-8 EFT Operatoren beschrieben werden kann im $W^{\pm}Z$ festzulegen und EFT Limits für diese Resonanzen zu bestimmen.



% Inhaltsverzeichnis
\tableofcontents
\printnoidxglossary[type=acronym]
\printacronyms

\pagenumbering{arabic}
\chapter{Introduction}
\subfile{sections/Introduction.tex}

\chapter{Experimental Setup}
\subfile{sections/ATLAS.tex}

\chapter{Theoretical Foundation}
\label{chapter:Theory}
\section{Effective Field Theory}
\subfile{sections/EFT.tex}

\section{BSM Theories}
\subfile{sections/BSM.tex}

\section{Vector Boson Scattering}
\subfile{sections/VBS.tex}

\chapter{Event Selection and Statistical Method}
\label{chapter:Event_and_stat}
\subfile{sections/Event_selection.tex}

\chapter{Results}
\subfile{sections/dim_8_gm.tex}

\chapter{Summery and Outlook}
\subfile{sections/summery.tex}


\newpage
\nocite{*}
\printbibliography

\listoffigures
\listoftables

% Erklärung
\clearpage
\thispagestyle{empty}
\minisec{Erklärung}\vspace*{1.5em}

Hiermit erkläre ich, dass ich diese Arbeit im Rahmen der Betreuung am Institut
für ??? Physik ohne unzulässige Hilfe Dritter verfasst und alle Quellen als solche gekennzeichnet habe.

\vspace*{45em}

Georg Schmieder \par
Dresden, Mai 2022

\chapter{Appendix}
\label{sec:appendix}
\subfile{sections/appendix.tex}

\end{document}