\documentclass[../Bachelorarbeit.tex]{subfiles}

\begin{document}
CERN may be the most famous scientific institution in the World. The Large Hadron Collider(LHC) build in 2008 is the largest circular collider in the World.
Famous for the first discovery of the Higgs-Boson the two-ring-superconducting-hadron accelerator and collider are where in a 26.7 km tunnel.
This Technological marble is only possible because of thousands of engineers, computer scientists and physicists.
Four large detectors ALICE, ATLAS, CMS and LHC and other smaller experiments are connected to the LHC.
The LHC aims for a collision energy of 14 TeV as of now Run-II achieved 13 TeV by accelerating particle beams up to 6.5 TeV.
The Number of events per second N is given by:
\begin{equation}
    \frac{dN}{dt} = L \sigma
\end{equation}

$\sigma$ is the cross-section of particles and L the luminosity defined as:

\begin{equation}
    L= \frac{N_{b}^{2} n_{b} f \gamma}{5 \pi \epsilon_{n} \beta^{*}} F \qquad \textnormal{ with } \qquad  F=\left( 1 + \left( \frac{\theta_{c} \sigma_{z}}{2 \sigma^{*}}\right)^{2} \right)^{-\frac{1}{2}}
\end{equation}
\begin{description}
    \item[$N_{b}$] number of particles per bunch
    \item[$n_{b}$]  number of bunches
    \item[$\epsilon_{n}$]  normalized transverse beam emittance
    \item[$\beta^{*}$] beta function at collision point
    \item[$F$] Luminosity correction in the interaction point based on the crossing angle
\end{description}
With the start of Run-3 on April 22, 2022 a collision energy of 13.6 TeV is expected.

\section{ATLAS}
ATLAS the largest general-purpose detector is a part of the LHC experiments with more than 5500 scientists working on
ATLAS experiments worldwide.


\textbf{Magnet System}\\
The Magnet System creates the Magnetic field that bends the trajectory of the incoming particles in order to measure the momentum of said particles.
For this purpose two superconducting Toroide magnets create a magnetic field $B=2$ T. The momentum can be calculated from the curvature $r$ due to the Lorenz force.
\begin{equation}
    r=\frac{p}{q B}
\end{equation}
where $p=\gamma m v$ is the relativistic momentum and q the charge of the particle. High momentum-particles resulte in small curvature while low-momentum particles show a grater curvature.
\\\\
\textbf{Inner Detector}\\
The inner detector consists of three Subdetectors. The ATLAS-Pixeldetector is the innermost part surrounded by the Semiconductor Tracker(SCT) providing additional data for the curvature calculation.
The outermost layer is the Transition Radiation Tracker (TRT). Transition radiation is to differentiate between fermions and hadrons.
\\\\
\textbf{Calorimeters}
The electromagnetic Calorimeter and the Hadron calorimeter make up the calorimeter system. Particels that interactions via electromagnetic force are measured in
the electromagnetic calorimeter. Electrons or photons create particle showers in the electromagnetic calorimeter and no showers in the hadronic calorimeter while hadrons show traces in both and showers mostly in the hadronic calorimeter.
This happens because hadrons have a higher mass and the cross-section ins inverse proportional to the mass of the particle.
\\\\

\textbf{Myon Spectrometer}
\end{document}
