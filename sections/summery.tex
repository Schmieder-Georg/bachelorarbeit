\documentclass[../Bachelorarbeit.tex]{subfiles}

\begin{document}
\label{sec:summery}
In this thesis the set out goal was to find the lowest energy resonance in $W^{\pm}Z$ scattering process that can still be described using \acrshort{eft}.
The results show that resonances with energy $E>800$ GeV and an estimated minimum cross-section $\sigma_{resonance}$ of $184$ ab can be described accurately.
The limits for all relevant parameters S0, M0, M1, T0, T1, T2 for the research mass points with greater than 800 GeV are shown in table \ref{tab:inv_mass_EFT_limits}.
These results are independent of the cross-section if the cross-section is greater than the minimum cross-section.\\
This result is only valid for the sensitivity of dim-8 operates in $W^{\pm}Z$ process, other \acrshort{vbs} process may produce a different result.
As the results are highly depend on binning further research with different binning depending on the resonance energy may be necessary since the same binning was used for all results.
Only statistical uncertainties where used in this study including systematic uncertainties is needed for a more accurate prediction for the sensitivity of dim-8 operators.
In this thesis dim-6 operates are assumed to be negligible or taken from another source which is often the case in \acrshort{eft} studies.
This however may not always be true and needs further research the sensitivity of a combination of dim-6 and dim-8 operators for low energy resonances can not be predicted using these results.
A next step in determining the sensitivity of dim-8 operators in \acrshort{vbs} is to look at other \acrshort{vbs} process which include different relevant operators.
Another interesting research would be to use different models than the \acrshort{gm} Model. \acrshort{eft} is model independent. The importance of matching the results to a \acrshort{bsm} Model can not be overstated.
Using other models will help to compare the models and if a \acrshort{bsm} interaction is found using \acrshort{eft} help matching the result to a \acrshort{uv} theory.
\end{document}