\documentclass[../Bachelorarbeit.tex]{subfiles}

\begin{document}
\label{sec:BSM}
\begin{figure}
    \centering
    \includegraphics[width=0.5\textwidth]{images/Teilchenwet_WW_range.png}
    \caption{Interaction strength for different ranges of the Fundamental interactions. Although the Nuclear force isn't a fundamental it is shown in order to visualize point (c). Translation of the interaction }
    \label{fig:WW_range}
\end{figure}

In 2012 the last particle predicted by the SM was detected at CERN the Higgs Boson. Yet again proving the success of the SM.
But this also means no more free parameters in the SM for new particles. All interactions found obey the local $SU(3) \times SU(2) \times U(1)$ gauge symmetries
and later data only strengthens the SM prediction. While no data was found suggesting inconsistencies with the electroweak symmetry breaking $SU(2) \times U(1) \rightarrow U(1)$.
This means physicist have to search for new interactions considering different interaction ranges and strengths(Figure \ref{fig:WW_range}).
\begin{enumerate}[label=(\alph*)]
    \item Considerably weaker than gravitation with infinite range
    \item Shorter range than the weak interaction of any strength
    \item Range between weak interaction and nuclear force and considerably weaker than the weak interaction 
\end{enumerate}

There are various methods for finding BSM particles and as many theoretical models for new interactions, but these can generally be broken down in three categories.
\\\\
The Model specific search where one takes a well-defined model often describing a complete UV theory and try's to find the predicted particles in measurements.
The most popular example for this is the Supersymmetry which is often associated with the search for Dark Matter and adds a wide range of different particles.
In the search for Dark Matter one would look for example look for missing transverse momentum in top quark interactions or look for resonance with the so-called
two Higgs doublet models.
\\\\
Another way of looking for new physics is by using simplified Models. Here one takes a well-defined model in order to describe some aspects or specific phenomenon of the UV Theory.
Again a good example comes from dark matter physics the beautifully named neutralino in Minimale Supersymmetry Standard Model(MSSM) with MSSM being a low energy model of the Supersymmetry.
Here the for neutralinos are electrically neutral fermions with a mass over 300 GeV and conserves the hypothetical R-parity in MSSM.
\\\\



\end{document}