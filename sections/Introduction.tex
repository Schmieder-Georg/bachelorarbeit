\documentclass[../Bachelorarbeit.tex]{subfiles}

\begin{document}
The goal of all physics around the world is to understand nature. From the Universe itself and huge objects like black holes described by Einstein's General Theory of relativity down to the smallest objects the fundamental particles describe by the standard model (\acrshort{sm}) of particle physics.
The vast difference is not only size but also in time. The cosmological timescale is often millions of years while particle physics work with life times as low as $10{-25}$s.
All cases come with their own unique challenges and obstacles, but in a sense all are equal all are described by physics. Combing the different theory is however not trivial.
The general relativity describes gravity but the quantum field theory and subsequently the \acrshort{sm} doesn't include a particle that describes gravity. The search for a theory beyond the \acrshort{sm} is an important aspect of modern physics.
At the \acrfull{lhc} measurement for most aspect of particle physics are conducted. One of the aspects is the scattering of $W^{\pm }$ and the $Z$ Boson and there interaction with the Higgs-Boson.
This thesis focuses on the search for beyond the standard model particles in the $W^{\pm}Z$ scattering process using the \acrfull{eft} approach.
EFT is often used for resonances with high energy compared to the invariant mass of the researched process. While in this thesis looks at low energy resonances with the main result being the lowest resonance energy possible that can be described by \acrshort{eft}.

Chapter \ref{sec:LHC} focuses on the \acrfull{lhc} and the ATLAS experiment.
While chapter \ref{chapter:Theory} gives an introduction in to effective field theory, beyond the standard model physics and the $W^{\pm}Z$ vector boson scattering process. The event selection and the statistics used to analyze the data are discussed in chapter \ref{chapter:Event_and_stat}.
In chapter \ref{sec:results} the results of the analysis are presented. The results discussed the \acrshort{eft} coefficients for different resonance masses and different cross-sections.
Lastly chapter \ref{sec:summery} summarizes the results and gives an outlook for further research.

\end{document}
